\documentclass{article}
\usepackage{algorithm}
\usepackage{algorithmic}
\usepackage{amsmath} 

\begin{document}

\renewcommand{\thealgorithm}{}
\begin{algorithm}
\caption{Causal mediation analysis for one data tuple}
\begin{algorithmic}
\STATE \textbf{Data:} Aligned and misaligned input data \textit{data\_tuple}, Model \textit{model}, Layer Number \textit{depth}, Head Number \textit{heads}, Embedding Dimension \textit{k}, Sequence Length \textit{t}
\STATE \textbf{Initialization:} Set model to evaluation mode, Initialize variables for storing results
\STATE \textbf{Procedure:}
\begin{itemize}
    \item Run model on aligned and misaligned data to get baseline probabilities, attention outputs and feedforward outputs
    \item Calculate the total effect by comparing the aligned and misaligned baseline probabilities
    \item Store the total effect
    \FOR{mediator \textbf{in} mediators (model, depth, heads, k, t)}
        \begin{itemize}
            \item Run model on misaligned data with the mediator output fixed to its output on aligned data
            \item Calculate the direct effect by comparing the aligned baseline probabilities with the probabilities obtained from the adapted run
            \item Run model on aligned data with the mediator output fixed to its output on misaligned data
            \item Calculate the indirect effect by comparing the aligned baseline probabilities with the probabilities obtained from the adapted run
            \item Store the direct and indirect effects
        \end{itemize}
    \ENDFOR
\end{itemize}

\STATE \textbf{Return:} Total, direct and indirect effects of mediators for the specified data tuple
\end{algorithmic}
\end{algorithm}

\end{document}